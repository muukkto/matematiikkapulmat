\clearpage

\section{Taylorin laskinpulma}

\begin{wrapfigure}{r}{0.3\textwidth}
    \begin{center}
    \includegraphics[width=0.28\textwidth]{kuvat/taylor_monsteri.jpg}
    \end{center}
\end{wrapfigure}

Monsteri Taylorilla on hirveiden, kauheiden ja karmivien esineiden sekatavarakauppa. Taylorilla on kuitenkin ongelma, koska hänen superlaskimensa on mennyt rikki. Kaupan nurkasta löytyi onneksi vanha ja pölyinen laskukone. Laskukoneella pystyy kuitenkin tarkastelemaan pelkästään polynomiyhtälöitä. 

\begin{enumerate}
\item Taylorin täytyisi laskea taajusgeneraattorin ominaisuuksia. Ongelmana on kuitenkin se, että generaattori muodostaa funktion $f(x) = \cos(x)$ muotoista vaihtovirtaa. Pystytkö auttamaan Tayloria ja muodostamaan polynomifunktion, joka approksimoi funktiota $f(x) = \cos(x)$ pisteen $x = 0$ ympäristössä?

\item Taylorin pitäisi myös laskea eksponentiaalisesti lisääntyvien myrkkykärpästen määrä. Pystytkö auttamaan Tayloria ja muodostamaan polynomifunktion, joka approksimoi funktiota $g(x) = e^x$ pisteen $x = 0$ ympäristössä?
\end{enumerate}

Vihje 1: Polynomifunktio on muotoa $a_nx^n+\ldots+a_2x^2+a_1x+a_0$

Vihje 2: Polynomifunktiota kannattaa lähteä rakentamaan vakiotermistä $a_0$. Tämän jälkeen ensimmäisen asteen termille pitäisi saada \textbf{kulmakerroin} $a_1$.


Vihje 3: Mitäköhän Taylorin sarja mahtaa tehdä?